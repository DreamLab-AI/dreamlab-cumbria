\documentclass{businessdocs}

\title{Risk Management and Mitigation Framework}
\author{Risk Management Committee}
\date{\today}

\usepackage{rotating}
\usepackage{pgf-pie}

\begin{document}

\maketitle
\tableofcontents
\newpage

\begin{executivesummary}
This document establishes a comprehensive risk management framework for the Business Excellence Initiative. It identifies key risk categories, assessment methodologies, mitigation strategies, and monitoring procedures to ensure business continuity and protect stakeholder interests. The framework follows ISO 31000 principles adapted for our specific business context.
\end{executivesummary}

\section{Risk Management Overview}

\subsection{Risk Management Philosophy}

Our approach to risk management is based on the following principles:
\begin{itemize}
    \item Proactive identification and assessment of risks
    \item Integration of risk management into all business processes
    \item Clear ownership and accountability for risk mitigation
    \item Regular monitoring and reporting of risk status
    \item Continuous improvement of risk management practices
\end{itemize}

\subsection{Risk Governance Structure}

\begin{figure}[H]
\centering
\begin{tikzpicture}[node distance=2cm]
    % Board level
    \node[rectangle,draw=corporate-blue,fill=corporate-lightblue!30,minimum width=12cm,minimum height=1.5cm] (board) {Board of Directors - Ultimate Risk Oversight};
    
    % Executive level
    \node[rectangle,draw=corporate-blue,fill=corporate-lightblue!30,minimum width=6cm,minimum height=1.5cm,below of=board,xshift=-3cm] (exec) {Executive Committee};
    \node[rectangle,draw=corporate-blue,fill=corporate-lightblue!30,minimum width=6cm,minimum height=1.5cm,below of=board,xshift=3cm] (audit) {Audit Committee};
    
    % Risk committee
    \node[rectangle,draw=accent-orange,fill=accent-orange!20,minimum width=10cm,minimum height=1.5cm,below of=board,yshift=-3cm] (riskcom) {Risk Management Committee};
    
    % Operational level
    \node[rectangle,draw=corporate-grey,fill=corporate-lightgrey,minimum width=3cm,minimum height=1.2cm,below of=riskcom,xshift=-4cm,yshift=-1cm] (tech) {Technology};
    \node[rectangle,draw=corporate-grey,fill=corporate-lightgrey,minimum width=3cm,minimum height=1.2cm,below of=riskcom,xshift=-1.3cm,yshift=-1cm] (ops) {Operations};
    \node[rectangle,draw=corporate-grey,fill=corporate-lightgrey,minimum width=3cm,minimum height=1.2cm,below of=riskcom,xshift=1.3cm,yshift=-1cm] (finance) {Finance};
    \node[rectangle,draw=corporate-grey,fill=corporate-lightgrey,minimum width=3cm,minimum height=1.2cm,below of=riskcom,xshift=4cm,yshift=-1cm] (market) {Marketing};
    
    % Connections
    \draw[->,thick] (board) -- (exec);
    \draw[->,thick] (board) -- (audit);
    \draw[->,thick] (exec) -- (riskcom);
    \draw[->,thick] (audit) -- (riskcom);
    \draw[->,thick] (riskcom) -- (tech);
    \draw[->,thick] (riskcom) -- (ops);
    \draw[->,thick] (riskcom) -- (finance);
    \draw[->,thick] (riskcom) -- (market);
\end{tikzpicture}
\caption{Risk governance hierarchy}
\end{figure}

\section{Risk Assessment Methodology}

\subsection{Risk Scoring Matrix}

We use a 5x5 risk matrix to assess and prioritise risks based on likelihood and impact:

\begin{figure}[H]
\centering
\begin{tikzpicture}
    % Grid
    \draw[thick] (0,0) grid (5,5);
    
    % Colour coding
    \fill[accent-green!30] (0,0) rectangle (2,2);
    \fill[accent-orange!30] (2,0) rectangle (5,2);
    \fill[accent-orange!30] (0,2) rectangle (2,5);
    \fill[accent-red!30] (2,2) rectangle (5,5);
    
    % Labels
    \node at (-0.5,2.5) [rotate=90] {\textbf{Likelihood}};
    \node at (2.5,-0.5) {\textbf{Impact}};
    
    % Scale labels
    \node at (-0.5,0.5) {\small 1};
    \node at (-0.5,1.5) {\small 2};
    \node at (-0.5,2.5) {\small 3};
    \node at (-0.5,3.5) {\small 4};
    \node at (-0.5,4.5) {\small 5};
    
    \node at (0.5,-0.5) {\small 1};
    \node at (1.5,-0.5) {\small 2};
    \node at (2.5,-0.5) {\small 3};
    \node at (3.5,-0.5) {\small 4};
    \node at (4.5,-0.5) {\small 5};
    
    % Legend
    \node[rectangle,fill=accent-green!30,minimum width=1cm,minimum height=0.5cm] at (7,4) {};
    \node[right] at (7.5,4) {\small Low Risk (1-4)};
    
    \node[rectangle,fill=accent-orange!30,minimum width=1cm,minimum height=0.5cm] at (7,3) {};
    \node[right] at (7.5,3) {\small Medium Risk (5-12)};
    
    \node[rectangle,fill=accent-red!30,minimum width=1cm,minimum height=0.5cm] at (7,2) {};
    \node[right] at (7.5,2) {\small High Risk (15-25)};
    
    % Risk score formula
    \node at (2.5,-2) {\textbf{Risk Score = Likelihood × Impact}};
\end{tikzpicture}
\caption{Risk assessment matrix}
\end{figure}

\subsection{Risk Categories}

\begin{table}[H]
\centering
\begin{tabular}{@{}llp{8cm}@{}}
\toprule
\textbf{Category} & \textbf{Code} & \textbf{Description} \\
\midrule
Strategic & STR & Risks affecting long-term objectives and market position \\
Operational & OPS & Risks in day-to-day operations and service delivery \\
Financial & FIN & Risks affecting financial performance and liquidity \\
Compliance & COM & Regulatory and legal compliance risks \\
Technology & TEC & IT infrastructure, cybersecurity, and platform risks \\
Reputational & REP & Risks affecting brand value and stakeholder trust \\
\bottomrule
\end{tabular}
\caption{Risk category definitions}
\end{table}

\section{Risk Register}

\subsection{Critical Risks}

\begin{sidewaystable}
\centering
\small
\begin{tabular}{@{}llccccll@{}}
\toprule
\textbf{ID} & \textbf{Risk Description} & \textbf{Category} & \textbf{Likelihood} & \textbf{Impact} & \textbf{Score} & \textbf{Mitigation Strategy} & \textbf{Owner} \\
\midrule
STR-001 & New competitor with superior technology & Strategic & 3 & 4 & 12 & Continuous innovation, patent protection & CTO \\
STR-002 & Market demand shifts away from online learning & Strategic & 2 & 5 & 10 & Diversification, hybrid offerings & CEO \\
OPS-001 & Key talent departure & Operational & 4 & 3 & 12 & Retention programmes, succession planning & CHRO \\
OPS-002 & Service outage affecting users & Operational & 2 & 4 & 8 & Redundancy, disaster recovery & CTO \\
FIN-001 & Funding round fails or delayed & Financial & 3 & 5 & 15 & Multiple funding sources, cost control & CFO \\
FIN-002 & Customer acquisition costs exceed targets & Financial & 4 & 3 & 12 & Channel optimisation, organic growth & CMO \\
COM-001 & Data protection regulation breach & Compliance & 2 & 5 & 10 & GDPR compliance, regular audits & Legal \\
TEC-001 & Cybersecurity breach & Technology & 3 & 5 & 15 & Security protocols, insurance & CTO \\
TEC-002 & AI algorithm bias complaints & Technology & 3 & 4 & 12 & Algorithm audits, transparency & CTO \\
REP-001 & Negative media coverage & Reputational & 3 & 3 & 9 & PR strategy, crisis management & CMO \\
\bottomrule
\end{tabular}
\caption{Critical risk register}
\end{sidewaystable}

\subsection{Risk Heat Map}

\begin{figure}[H]
\centering
\begin{tikzpicture}
    % Grid
    \draw[thick] (0,0) grid (5,5);
    
    % Colour coding
    \fill[accent-green!30] (0,0) rectangle (2,2);
    \fill[accent-orange!30] (2,0) rectangle (5,2);
    \fill[accent-orange!30] (0,2) rectangle (2,5);
    \fill[accent-red!30] (2,2) rectangle (5,5);
    
    % Plot risks
    \node[circle,fill=corporate-blue,minimum size=8mm] at (2.5,4.5) {\tiny FIN-001};
    \node[circle,fill=corporate-blue,minimum size=8mm] at (2.5,4.5) {\tiny TEC-001};
    \node[circle,fill=corporate-blue,minimum size=8mm] at (3,3.5) {\tiny STR-001};
    \node[circle,fill=corporate-blue,minimum size=8mm] at (3.5,2.5) {\tiny FIN-002};
    \node[circle,fill=corporate-blue,minimum size=8mm] at (3.5,2.5) {\tiny OPS-001};
    \node[circle,fill=corporate-blue,minimum size=8mm] at (2.5,3.5) {\tiny TEC-002};
    \node[circle,fill=corporate-blue,minimum size=8mm] at (1.5,4.5) {\tiny STR-002};
    \node[circle,fill=corporate-blue,minimum size=8mm] at (1.5,4.5) {\tiny COM-001};
    \node[circle,fill=corporate-blue,minimum size=8mm] at (1.5,3.5) {\tiny OPS-002};
    \node[circle,fill=corporate-blue,minimum size=8mm] at (2.5,2.5) {\tiny REP-001};
    
    % Axes
    \node at (-0.5,2.5) [rotate=90] {\textbf{Likelihood}};
    \node at (2.5,-0.5) {\textbf{Impact}};
\end{tikzpicture}
\caption{Current risk heat map}
\end{figure}

\section{Mitigation Strategies}

\subsection{Strategic Risk Mitigation}

\begin{riskbox}
\textbf{STR-001: Competitive Threat Mitigation}
\begin{enumerate}
    \item \textbf{Innovation Pipeline}: Maintain 6-month feature roadmap ahead of competition
    \item \textbf{IP Protection}: File patents for core AI algorithms
    \item \textbf{Strategic Partnerships}: Exclusive content deals with industry leaders
    \item \textbf{Customer Lock-in}: Annual contracts with switching costs
    \item \textbf{Market Intelligence}: Continuous competitor monitoring
\end{enumerate}

\textbf{Success Metrics}:
\begin{itemize}
    \item Feature release velocity: 2 major features/month
    \item Patent applications: 4 per year
    \item Partnership agreements: 10+ exclusive deals
    \item Customer retention: >90\% annual
\end{itemize}
\end{riskbox}

\subsection{Financial Risk Mitigation}

\begin{table}[H]
\centering
\begin{tabular}{@{}ll@{}}
\toprule
\textbf{Risk Area} & \textbf{Mitigation Approach} \\
\midrule
Funding dependency & \begin{tabular}[t]{@{}l@{}}
• Maintain 12-month cash runway\\
• Develop revenue-based financing options\\
• Build relationships with 20+ investors\\
• Achieve cash flow positive by Month 18
\end{tabular} \\
\midrule
CAC management & \begin{tabular}[t]{@{}l@{}}
• Diversify acquisition channels\\
• Invest in organic/content marketing\\
• Implement referral programme\\
• Target CAC:CLV ratio of 1:4
\end{tabular} \\
\midrule
Currency exposure & \begin{tabular}[t]{@{}l@{}}
• Natural hedging through costs\\
• Forward contracts for major exposures\\
• Multi-currency pricing strategy\\
• Regular FX risk assessment
\end{tabular} \\
\bottomrule
\end{tabular}
\caption{Financial risk mitigation strategies}
\end{table}

\subsection{Technology Risk Mitigation}

\begin{figure}[H]
\centering
\begin{tikzpicture}[node distance=2.5cm]
    % Central security
    \node[circle,draw=accent-green,fill=accent-green!20,minimum size=3cm,align=center] (center) {\textbf{Security}\\Core};
    
    % Security layers
    \node[rectangle,draw=corporate-blue,fill=corporate-lightgrey,above of=center] (prevent) {Prevention\\WAF, DDoS Protection};
    \node[rectangle,draw=corporate-blue,fill=corporate-lightgrey,right of=center] (detect) {Detection\\SIEM, Monitoring};
    \node[rectangle,draw=corporate-blue,fill=corporate-lightgrey,below of=center] (respond) {Response\\Incident Team};
    \node[rectangle,draw=corporate-blue,fill=corporate-lightgrey,left of=center] (recover) {Recovery\\Backups, DR};
    
    % Connections
    \draw[->,thick] (center) -- (prevent);
    \draw[->,thick] (center) -- (detect);
    \draw[->,thick] (center) -- (respond);
    \draw[->,thick] (center) -- (recover);
    
    % Cycle arrows
    \draw[->,thick,dashed] (prevent) to[bend left=30] (detect);
    \draw[->,thick,dashed] (detect) to[bend left=30] (respond);
    \draw[->,thick,dashed] (respond) to[bend left=30] (recover);
    \draw[->,thick,dashed] (recover) to[bend left=30] (prevent);
\end{tikzpicture}
\caption{Cybersecurity defence model}
\end{figure}

\section{Risk Monitoring and Reporting}

\subsection{Key Risk Indicators (KRIs)}

\begin{table}[H]
\centering
\begin{tabular}{@{}llccc@{}}
\toprule
\textbf{Risk Area} & \textbf{KRI} & \textbf{Green} & \textbf{Amber} & \textbf{Red} \\
\midrule
Financial & Cash runway (months) & >12 & 6-12 & <6 \\
Financial & CAC:CLV ratio & <1:4 & 1:3-1:4 & >1:3 \\
Operational & Platform uptime & >99.9\% & 99.5-99.9\% & <99.5\% \\
Operational & Employee turnover & <10\% & 10-20\% & >20\% \\
Customer & NPS score & >50 & 30-50 & <30 \\
Customer & Monthly churn & <3\% & 3-5\% & >5\% \\
Technology & Security incidents & 0 & 1-2 & >2 \\
Compliance & Audit findings & 0 & 1-3 & >3 \\
\bottomrule
\end{tabular}
\caption{Key risk indicators and thresholds}
\end{table}

\subsection{Risk Reporting Dashboard}

\begin{figure}[H]
\centering
\begin{tikzpicture}
    % Overall risk status
    \node[rectangle,draw=corporate-blue,fill=corporate-lightblue!30,minimum width=12cm,minimum height=2cm] at (0,5) {
        \Large\textbf{Overall Risk Status: MEDIUM}\\
        \small Last Updated: \today
    };
    
    % Risk category status boxes
    \node[rectangle,draw=accent-green,fill=accent-green!20,minimum width=3.5cm,minimum height=2cm] at (-4,2) {
        \textbf{Strategic}\\
        \small 2 High, 3 Medium\\
        Trend: Stable
    };
    
    \node[rectangle,draw=accent-orange,fill=accent-orange!20,minimum width=3.5cm,minimum height=2cm] at (0,2) {
        \textbf{Financial}\\
        \small 1 High, 4 Medium\\
        Trend: Improving
    };
    
    \node[rectangle,draw=accent-red,fill=accent-red!20,minimum width=3.5cm,minimum height=2cm] at (4,2) {
        \textbf{Technology}\\
        \small 2 High, 2 Medium\\
        Trend: Worsening
    };
    
    % Action items
    \node[rectangle,draw=corporate-grey,fill=corporate-lightgrey,minimum width=10cm,minimum height=2cm] at (0,-1) {
        \textbf{Critical Actions Required}\\
        \small • Complete Series A funding by Q2\\
        \small • Implement enhanced security measures\\
        \small • Finalise key hire succession plans
    };
\end{tikzpicture}
\caption{Executive risk dashboard template}
\end{figure}

\section{Crisis Management}

\subsection{Crisis Response Framework}

\begin{table}[H]
\centering
\begin{tabular}{@{}lp{10cm}@{}}
\toprule
\textbf{Phase} & \textbf{Actions} \\
\midrule
Detection (0-1 hour) & \begin{itemize}
    \item Incident identification and initial assessment
    \item Activate crisis management team
    \item Implement immediate containment measures
    \item Begin documentation of events
\end{itemize} \\
\midrule
Assessment (1-4 hours) & \begin{itemize}
    \item Full impact assessment
    \item Stakeholder identification
    \item Resource mobilisation
    \item Initial communications draft
\end{itemize} \\
\midrule
Response (4-24 hours) & \begin{itemize}
    \item Execute response plan
    \item Stakeholder communications
    \item Media management
    \item Continuous monitoring
\end{itemize} \\
\midrule
Recovery (1-7 days) & \begin{itemize}
    \item Restore normal operations
    \item Follow-up communications
    \item Lessons learned documentation
    \item Process improvements
\end{itemize} \\
\bottomrule
\end{tabular}
\caption{Crisis response phases}
\end{table}

\subsection{Communication Protocols}

\begin{figure}[H]
\centering
\begin{tikzpicture}[node distance=2cm]
    % Crisis team
    \node[circle,draw=accent-red,fill=accent-red!20,minimum size=2.5cm,align=center] (crisis) {\textbf{Crisis}\\Team};
    
    % Stakeholders
    \node[rectangle,draw=corporate-blue,fill=corporate-lightgrey,above left of=crisis,xshift=-1cm] (board) {Board};
    \node[rectangle,draw=corporate-blue,fill=corporate-lightgrey,above right of=crisis,xshift=1cm] (investors) {Investors};
    \node[rectangle,draw=corporate-blue,fill=corporate-lightgrey,right of=crisis,xshift=1cm] (customers) {Customers};
    \node[rectangle,draw=corporate-blue,fill=corporate-lightgrey,below right of=crisis,xshift=1cm] (employees) {Employees};
    \node[rectangle,draw=corporate-blue,fill=corporate-lightgrey,below left of=crisis,xshift=-1cm] (media) {Media};
    \node[rectangle,draw=corporate-blue,fill=corporate-lightgrey,left of=crisis,xshift=-1cm] (regulators) {Regulators};
    
    % Communication lines
    \draw[<->,thick,accent-red] (crisis) -- (board);
    \draw[<->,thick,accent-red] (crisis) -- (investors);
    \draw[<->,thick,accent-red] (crisis) -- (customers);
    \draw[<->,thick,accent-red] (crisis) -- (employees);
    \draw[<->,thick,accent-red] (crisis) -- (media);
    \draw[<->,thick,accent-red] (crisis) -- (regulators);
\end{tikzpicture}
\caption{Crisis communication structure}
\end{figure}

\section{Business Continuity Planning}

\subsection{Critical Business Functions}

\begin{table}[H]
\centering
\begin{tabular}{@{}llcc@{}}
\toprule
\textbf{Function} & \textbf{Priority} & \textbf{RTO*} & \textbf{RPO**} \\
\midrule
Learning platform & Critical & 2 hours & 1 hour \\
Payment processing & Critical & 4 hours & 2 hours \\
Customer support & High & 8 hours & 4 hours \\
Content delivery & High & 4 hours & 2 hours \\
Marketing systems & Medium & 24 hours & 12 hours \\
Internal systems & Low & 48 hours & 24 hours \\
\bottomrule
\end{tabular}
\caption{Business function recovery priorities}
\small *RTO: Recovery Time Objective, **RPO: Recovery Point Objective
\end{table}

\subsection{Scenario Planning}

\begin{opportunitybox}
\textbf{Prepared Scenarios}:
\begin{enumerate}
    \item \textbf{Technology Failure}: Complete platform outage
    \item \textbf{Data Breach}: Customer data compromise
    \item \textbf{Key Person Loss}: CEO/CTO sudden departure
    \item \textbf{Funding Crisis}: Unable to raise next round
    \item \textbf{Regulatory Change}: New education regulations
    \item \textbf{Pandemic 2.0}: Global business disruption
\end{enumerate}

Each scenario has detailed response plans including:
\begin{itemize}
    \item Immediate actions checklist
    \item Communication templates
    \item Resource allocation plans
    \item Recovery timelines
\end{itemize}
\end{opportunitybox}

\section{Risk Culture and Training}

\subsection{Risk Awareness Programme}

\begin{figure}[H]
\centering
\begin{tikzpicture}
    \begin{axis}[
        width=0.8\textwidth,
        height=6cm,
        xlabel={Quarter},
        ylabel={Training Completion (\%)},
        legend pos=north west,
        ymin=0,
        ymax=100,
        grid=major,
        grid style={dashed,gray!30}
    ]
    \addplot[color=corporate-blue,mark=*,thick] coordinates {
        (1,45) (2,65) (3,78) (4,85) (5,90) (6,92) (7,95) (8,98)
    };
    \addlegendentry{All Staff}
    
    \addplot[color=accent-green,mark=square*,thick] coordinates {
        (1,60) (2,80) (3,90) (4,95) (5,98) (6,100) (7,100) (8,100)
    };
    \addlegendentry{Management}
    \end{axis}
\end{tikzpicture}
\caption{Risk training completion rates}
\end{figure}

\subsection{Risk Management Maturity}

\begin{table}[H]
\centering
\begin{tabular}{@{}lcccc@{}}
\toprule
\textbf{Dimension} & \textbf{Current} & \textbf{Target} & \textbf{Gap} & \textbf{Actions} \\
\midrule
Risk identification & 3 & 4 & 1 & Automated scanning tools \\
Risk assessment & 3 & 4 & 1 & Quantitative modelling \\
Risk mitigation & 2 & 4 & 2 & Proactive strategies \\
Risk monitoring & 2 & 5 & 3 & Real-time dashboards \\
Risk culture & 3 & 5 & 2 & Training programme \\
\bottomrule
\end{tabular}
\caption{Risk management maturity assessment (1-5 scale)}
\end{table}

\section{Insurance and Risk Transfer}

\subsection{Insurance Coverage}

\begin{table}[H]
\centering
\begin{tabular}{@{}llr@{}}
\toprule
\textbf{Policy Type} & \textbf{Coverage} & \textbf{Limit} \\
\midrule
General Liability & Third-party claims & \pounds{5,000,000} \\
Professional Indemnity & Service failures & \pounds{10,000,000} \\
Cyber Insurance & Data breaches, attacks & \pounds{5,000,000} \\
Directors \& Officers & Management liability & \pounds{5,000,000} \\
Business Interruption & Revenue loss & \pounds{2,000,000} \\
Key Person & CEO/CTO life/disability & \pounds{5,000,000} \\
\bottomrule
\end{tabular}
\caption{Insurance portfolio summary}
\end{table}

\section{Appendices}

\subsection{Risk Assessment Templates}

Templates available for:
\begin{itemize}
    \item New project risk assessment
    \item Vendor risk evaluation
    \item Technology change risk review
    \item Regulatory compliance checklist
    \item Business continuity test scenarios
\end{itemize}

\subsection{Contact Information}

\begin{keypoint}
\textbf{Crisis Management Contacts}:
\begin{itemize}
    \item Crisis Hotline: +44 800 XXX XXXX (24/7)
    \item CEO: jane.smith@company.com (+44 7XXX XXXXXX)
    \item CRO: risk.officer@company.com (+44 7XXX XXXXXX)
    \item Legal: legal.team@company.com
    \item PR Agency: crisis@pragency.com
\end{itemize}
\end{keypoint}

\end{document}