\documentclass{businessdocs}

\title{Business Documentation Template Example}
\author{Documentation Team}
\date{\today}

\addbibresource{../bibliography/references.bib}

\begin{document}

\maketitle
\tableofcontents
\newpage

\begin{executivesummary}
This document demonstrates the features and capabilities of the custom business documentation class. It provides a consistent styling framework for all business documents, ensuring professional presentation and brand coherence across the entire documentation suite.
\end{executivesummary}

\section{Introduction}

This template provides comprehensive styling for business documents with UK English spelling throughout. The class includes custom environments, financial formatting, and professional visualisation capabilities.

\subsection{Key Features}

\begin{itemize}
    \item Consistent corporate branding with predefined colours
    \item Professional typography using TeX Gyre fonts
    \item Integrated bibliography management
    \item Custom environments for key points, risks, and opportunities
    \item Financial formatting commands
    \item TikZ/PGF integration for diagrams
    \item Cross-referencing support
\end{itemize}

\section{Using Custom Environments}

\begin{keypoint}
This is a key point box that highlights important information. It uses corporate colours and maintains consistent styling across all documents.
\end{keypoint}

\begin{riskbox}
Risk assessments can be highlighted using this specialised environment. The red accent colour draws attention to potential challenges.
\end{riskbox}

\begin{opportunitybox}
Opportunities are presented in green-accented boxes, providing visual distinction for positive aspects and growth potential.
\end{opportunitybox}

\section{Financial Formatting}

The template includes commands for properly formatting financial figures:

\begin{itemize}
    \item Revenue target: \pounds{2500000}
    \item European expansion: \euros{1800000}
    \item US market opportunity: \dollars{3200000}
\end{itemize}

\section{Visual Elements}

\subsection{KPI Display}

\begin{center}
\kpi{Customer Acquisition}{15,000}{Q1 2024 Target}
\quad
\kpi{Revenue Growth}{45\%}{Year-on-Year}
\quad
\kpi{Market Share}{12\%}{UK Market}
\end{center}

\subsection{Process Diagram}

\begin{figure}[H]
\centering
\begin{tikzpicture}[node distance=3cm]
    \node[rectangle,draw=corporate-blue,fill=corporate-lightgrey,minimum width=2.5cm,minimum height=1.5cm] (start) {Initiation};
    \node[rectangle,draw=corporate-blue,fill=corporate-lightgrey,minimum width=2.5cm,minimum height=1.5cm,right of=start] (plan) {Planning};
    \node[rectangle,draw=corporate-blue,fill=corporate-lightgrey,minimum width=2.5cm,minimum height=1.5cm,right of=plan] (exec) {Execution};
    \node[rectangle,draw=corporate-blue,fill=corporate-lightgrey,minimum width=2.5cm,minimum height=1.5cm,right of=exec] (close) {Closure};

    \draw[->,thick,corporate-blue] (start) -- (plan);
    \draw[->,thick,corporate-blue] (plan) -- (exec);
    \draw[->,thick,corporate-blue] (exec) -- (close);
\end{tikzpicture}
\caption{Project lifecycle phases}
\label{fig:lifecycle}
\end{figure}

\section{Cross-References}

The template supports intelligent cross-referencing. For example, \cref{fig:lifecycle} shows the project lifecycle phases. This functionality extends to all document elements including sections, tables, and equations.

\printbibliography

\end{document}